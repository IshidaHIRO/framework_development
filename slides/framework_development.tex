% Created 2014-10-28 火 19:07
\documentclass[t, aspectratio=169]{beamer}
\usepackage{zxjatype}
\usepackage[ipa]{zxjafont}
\setbeamertemplate{navigation symbols}{}
\hypersetup{colorlinks,linkcolor=,urlcolor=gray}
\AtBeginPart
{
  \begin{frame}<beamer|handout>
    \date{\insertpart}
    \maketitle
  \end{frame}
}
\AtBeginSection[]
{
  \begin{frame}<beamer>
  \tableofcontents[currentsection,currentsubsection]
  \end{frame}
}

\usepackage{minted}
\institute[AIIT]{産業技術大学院大学(AIIT)}
\usetheme{Berkeley}
\usecolortheme{seahorse}
\useinnertheme{rectangles}
\author{中鉢 欣秀}
\date{}
\title{フレームワーク開発特論}
\hypersetup{
  pdfkeywords={},
  pdfsubject={},
  pdfcreator={Emacs 24.3.2 (Org mode 8.2.5h)}}
\begin{document}

\maketitle

\part{第1章 この授業について}
\label{sec-1}
\section{導入}
\label{sec-1-1}
\begin{frame}[label=sec-1-1-1]{はじめに}
\begin{block}{自己紹介}
\begin{itemize}
\item \href{https://github.com/ychubachi/enpit/blob/master/slides/self_introduction.org}{enpit/self\_introduction.org at master · ychubachi/enpit}
\end{itemize}
\end{block}

\begin{block}{シラバス}
\begin{itemize}
\item \href{http://aiit.ac.jp/master_program/isa/lecture/pdf/h26/4_6.pdf}{フレームワーク開発特論}
\end{itemize}
\end{block}

\begin{block}{ハンドアウト}
\begin{itemize}
\item \href{https://github.com/ychubachi/framework_development/blob/master/slides/framework_development.org}{framework\_development/framework\_development.org at master · ychubachi/framework\_development}
\end{itemize}
\end{block}
\end{frame}

\begin{frame}[label=sec-1-1-2]{授業の計画}
\begin{block}{スケジュールの調整}
\begin{itemize}
\item 基本的にはシラバス通りに行うが,
本年度から内容を刷新して新規に行う授業のため
計画の変更がありえる
\end{itemize}
\end{block}

\begin{block}{休講・補講}
\begin{itemize}
\item 10/9(木) ビデオ視聴(休講)
\end{itemize}
\end{block}
\end{frame}
\section{学習目的・目標}
\label{sec-1-2}
\begin{frame}[label=sec-1-2-1]{この授業の目的}
\begin{itemize}
\item 再利用可能なコンポーネント開発の概念的理解を行う
\item RubyのGemを題材に,実装技術を学ぶ
\end{itemize}
\end{frame}

\begin{frame}[label=sec-1-2-2]{この授業の目標}
\begin{itemize}
\item Rubyで実際にコーデングを行い,コンポーネントを開発できる
\item Git/GitHubを用い,開発の効率の向上・成果物の公開などができる
\item Rubyのテスト技術について学ぶ
\item RubyGemsに公開する方法
\end{itemize}
\end{frame}
\section{授業の方法}
\label{sec-1-3}
\begin{frame}[label=sec-1-3-1]{この授業で用いる資料}
\begin{itemize}
\item この授業で取り上げる資料は,開発コミュニティが公開している
Webページを中心に適宜紹介する
\item エンジニアは英語の原典を読めなくてはならないので,英語のページ
を見ながら解説する
\item スライド・ハンドアウトは適宜更新するので,
必要に応じてダウンロードすること
\end{itemize}
\end{frame}
\begin{frame}[label=sec-1-3-2]{Git/GitHubの活用}
\begin{itemize}
\item GitHubのアカウントを作成しておくこと
\item ソースコードを作成する課題は,GitHubにも登録してもらうことがある
\end{itemize}
\end{frame}
\begin{frame}[label=sec-1-3-3]{仮想化環境の準備}
\begin{itemize}
\item Rubyの開発環境
\begin{itemize}
\item enPiTの仮想環境を利用
\end{itemize}
\item インストールの方法
\begin{itemize}
\item 資料
\begin{itemize}
\item \href{https://github.com/ychubachi/enpit/blob/master/slides/preparation.org}{enpit/preparation.org at master · ychubachi/enpit}
\end{itemize}
\item 動画を参照
\begin{itemize}
\item \url{http://youtu.be/kePqg8dCgJM}
\end{itemize}
\item 注意
\begin{itemize}
\item 動画の視聴及び作業のために2時間程度かかる
\end{itemize}
\end{itemize}
\end{itemize}
\end{frame}
\begin{frame}[fragile,label=sec-1-3-4]{仮想化環境の設定}
 \begin{block}{内容}
\begin{itemize}
\item enPiT用に作成したVagrantのboxファイルの入手して
実行できるようにする
\item Vagrantのバージョンは最新版にしておく
\item Vagrantのインストール後,次のコマンドでインストール可能
\end{itemize}
\end{block}
\begin{block}{コマンド}
\begin{minted}[]{bash}
vagrant init ychubachi/enpit
\end{minted}
\end{block}
\end{frame}
\section{RubyのGemとは}
\label{sec-1-4}
\begin{frame}[label=sec-1-4-1]{Rubyによるコンポーネント}
\begin{itemize}
\item Rubyには,再利用可能なコンポーネントを
取り扱う仕組みとしてGemがある
\item 資料
\begin{itemize}
\item \href{http://guides.rubygems.org/}{RubyGems Guides}
\end{itemize}
\end{itemize}
\end{frame}

\begin{frame}[fragile,label=sec-1-4-2]{基本的なコマンド}
 \begin{itemize}
\item \texttt{gem} コマンドは,Rubyでコンポーネントを開発したり,
配布をしたりする等の際に利用するコマンド
\item 後に解説する \texttt{bundler} コマンドの基盤
\item 資料
\begin{itemize}
\item \href{http://guides.rubygems.org/rubygems-basics/}{RubyGems Basics - RubyGems Guides}
\end{itemize}
\end{itemize}
\end{frame}

\section{演習課題}
\label{sec-1-5}
\begin{frame}[label=sec-1-5-1]{課題1-1}
\begin{block}{開発環境の構築}
\begin{itemize}
\item 演習用仮想化環境を用意する
\item 資料と動画を参照
\end{itemize}
\end{block}

\begin{block}{作業内容}
\begin{itemize}
\item VirtualBox と Vagrantをインストールする
\begin{itemize}
\item \href{https://www.virtualbox.org/}{Oracle VM VirtualBox}
\item \href{https://www.vagrantup.com/}{Vagrant}
\end{itemize}
\end{itemize}
\end{block}
\end{frame}

\begin{frame}[fragile,label=sec-1-5-2]{課題1-2}
 \begin{block}{開発環境の構築}
\begin{itemize}
\item enPiT仮想化環境をインストールする
\end{itemize}
\end{block}

\begin{block}{コマンド}
\begin{minted}[]{bash}
vagrant init ychubachi/enpit
vagrant up
vagrant ssh
\end{minted}
\end{block}
\end{frame}
\begin{frame}[fragile,label=sec-1-5-3]{課題1-3}
 \begin{block}{RubyGems Basics}
\begin{itemize}
\item 下記のガイドに記されたサンプルを実行し,
\texttt{gem} コマンドの基本的な使い方を学ぶ
\end{itemize}
\end{block}

\begin{block}{ガイド}
\begin{itemize}
\item \href{http://guides.rubygems.org/rubygems-basics/}{RubyGems Basics - RubyGems Guides}
\end{itemize}
\end{block}
\end{frame}

\begin{frame}[fragile,label=sec-1-5-4]{課題1-4}
 \begin{block}{演習}
\begin{itemize}
\item gemをダウンロードして中身を見てみる
\item \texttt{search}, \texttt{fetch},  \texttt{unpack} などのコマンドを活用する
\end{itemize}
\end{block}
\end{frame}

\part{第2章 RubyGemsの概要と周辺のツール群}
\label{sec-2}
\section{RubyGemsの解説(1)}
\label{sec-2-1}
\begin{frame}[label=sec-2-1-1]{Ruby Gemsのガイド}
\begin{itemize}
\item \href{http://guides.rubygems.org/what-is-a-gem/}{What is a gem? - RubyGems Guides}
\item \href{http://guides.rubygems.org/make-your-own-gem/}{Make your own gem - RubyGems Guides}
\item \href{http://guides.rubygems.org/gems-with-extensions/}{Gems with Extensions - RubyGems Guides}
\begin{itemize}
\item C言語拡張(省略)
\end{itemize}
\end{itemize}
\end{frame}

\section{RubyGemsの解説(2)}
\label{sec-2-2}
\begin{frame}[label=sec-2-2-1]{Ruby Gemsのガイド}
\begin{itemize}
\item \href{http://guides.rubygems.org/name-your-gem/}{Name your gem - RubyGems Guides}
\item \href{http://guides.rubygems.org/publishing/}{Publishing your gem - RubyGems Guides}
\item \href{http://guides.rubygems.org/security/}{Security - RubyGems Guides}
\begin{itemize}
\item セキュリティ(省略)
\end{itemize}
\item \href{http://guides.rubygems.org/patterns/}{Patterns - RubyGems Guides}
\end{itemize}
\end{frame}

\section{補足}
\label{sec-2-3}
\begin{frame}[label=sec-2-3-1]{MiniTest}
\begin{itemize}
\item \href{http://docs.ruby-lang.org/ja/2.0.0/library/minitest=2funit.html}{library minitest/unit}
\end{itemize}
\end{frame}

\section{演習課題}
\label{sec-2-4}
\begin{frame}[label=sec-2-4-1]{課題2-1 RubyGems.orgにアカウントを作成}
\begin{itemize}
\item RubyGems.orgにアカウントを作成しなさい
\item \href{https://rubygems.org/}{RubyGems.org | your community gem host}
\begin{itemize}
\item 「sign up」リンクから作成する
\end{itemize}
\end{itemize}
\end{frame}

\begin{frame}[label=sec-2-4-2]{課題2-2 ガイドを参考にGemを作る}
\begin{itemize}
\item ガイドの解説に従い,"hola" Gemを作成しなさい
\begin{itemize}
\item \href{http://guides.rubygems.org/make-your-own-gem/}{Make your own gem - RubyGems Guides}
\end{itemize}
\item 演習用Gemの名前の付け方
\begin{itemize}
\item hola\_(username)
\item 括弧内はRubyGemsのユーザ名に置き換えよ
\end{itemize}
\end{itemize}
\end{frame}

\part{第3章 Ruby自体のバージョン管理}
\label{sec-3}
\section{Rbenvの解説}
\label{sec-3-1}
\begin{frame}[label=sec-3-1-1]{Rubyのバージョン}
\begin{itemize}
\item Rubyには様々なバージョンがある
\begin{itemize}
\item 最新の安定版: Ruby 2.1.3
\item 前世代の安定版: Ruby 2.0.0-p576
\item 古い安定版: Ruby 1.9.3-p547
\item 1.8.7,1.9.2はサポート終了
\end{itemize}
\item これら以外にも,Javaや.NET Framework上で動作するものなど多数.
\end{itemize}
\end{frame}

\begin{frame}[label=sec-3-1-2]{開発時の混乱}
\begin{itemize}
\item 開発プロジェクトによって,異なるバージョンのRubyが用いられる
\item 複数の開発プロジェクトに参加する開発者が,
毎回手動でバージョンを変更するのは困難であるし,
トラブルの原因となる
\end{itemize}
\end{frame}

\begin{frame}[label=sec-3-1-3]{Rbenvについて}
\begin{block}{概要}
\begin{itemize}
\item バージョンの異なる複数のRubyを管理するツールであり,
主要なエコシステムの一部
\item プラグインを追加することで,インストールも自動化できる
\end{itemize}
\end{block}

\begin{block}{GitHub}
\begin{itemize}
\item \href{https://github.com/sstephenson/rbenv}{sstephenson/rbenv}
\end{itemize}
\end{block}

\begin{block}{RVMとの比較}
\begin{itemize}
\item \href{https://github.com/sstephenson/rbenv/wiki/Why-rbenv\%3F}{Why rbenv? · sstephenson/rbenv Wiki}
\end{itemize}
\end{block}
\end{frame}

\section{演習課題}
\label{sec-3-2}
\begin{frame}[fragile,label=sec-3-2-1]{課題3-1 別なRubyバージョンのインストール}
 \begin{itemize}
\item \texttt{rbenv} を用いて,異なるバージョンのRubyをインストールする
\end{itemize}
\end{frame}

\begin{frame}[fragile,label=sec-3-2-2]{課題3-2 Rubyのバージョンを切り替える}
 \begin{itemize}
\item \texttt{rbenv} を用い,=ruby= のバージョンを切り替える
\end{itemize}
\end{frame}

\part{第4章 プロジェクトで利用するGemの管理}
\label{sec-4}
\section{プロジェクトとGem}
\label{sec-4-1}
\begin{frame}[label=sec-4-1-1]{プロジェクトごとに異なるGemの集合}
\begin{itemize}
\item 開発プロジェクトにおいて利用するGemは異なるばかりではなく,
Gemのバージョンについても注意が必要である
\item Gemコマンドを直接用いてインストールする方法では,
必要なGemを主導で管理しなくてはならないし,
バージョンの異なるGemを用いる場合もある
\end{itemize}
\end{frame}

\section{Bundlerの解説}
\label{sec-4-2}
\begin{frame}[label=sec-4-2-1]{Bundlerについて}
\begin{itemize}
\item \href{http://bundler.io/}{Bundler: The best way to manage a Ruby application's gems}
\end{itemize}
\end{frame}

\begin{frame}[label=sec-4-2-2]{ドキュメントの解説}
\begin{itemize}
\item \href{http://bundler.io/rationale.html}{Why Bundler exists}
\item \href{https://github.com/sstephenson/rbenv/wiki/Understanding-binstubs}{Understanding binstubs · sstephenson/rbenv Wiki}
\item \href{http://bundler.io/gemfile.html}{Gemfile}
\end{itemize}
\end{frame}

\begin{frame}[fragile,label=sec-4-2-3]{bundleコマンドのインストール}
 \begin{block}{bundleコマンド}
\begin{itemize}
\item \texttt{gem} と違い, \texttt{bundle} コマンドは標準ではインストールされていない.
\end{itemize}
\end{block}

\begin{block}{Gemによるインストール方法}
\begin{minted}[]{bash}
gem install bundler
\end{minted}
\end{block}
\end{frame}
\section{BundlerによるGemの作成}
\label{sec-4-3}
\begin{frame}[fragile,label=sec-4-3-1]{ひな形の自動生成}
 \begin{block}{ひな形の自動生成}
\begin{itemize}
\item \href{http://bundler.io/v1.7/bundle_gem.html}{Bundler: The best way to manage a Ruby application's gems}
\end{itemize}
\end{block}

\begin{block}{コマンド}
\begin{minted}[]{bash}
bundle gem my_gem_name
\end{minted}
\end{block}
\end{frame}
\begin{frame}[fragile,label=sec-4-3-2]{雛形の内容}
 \begin{minted}[]{text}
.
├── .git
  <snip>
├── .gitignore
├── Gemfile
├── LICENSE.txt
├── README.md
├── Rakefile
├── lib
│   ├── my_gem_name
│   │   └── version.rb
│   └── my_gem_name.rb
└── my_gem_name.gemspec
\end{minted}
\end{frame}
\begin{frame}[fragile,label=sec-4-3-3]{\texttt{git ls-files} について}
 \begin{itemize}
\item gitにindexされているファイルの一覧
\item 新しいファイルは,ステージング領域にaddされると表示に加わる
\item \texttt{.gitignore} で無視するファイルを設定できる
\end{itemize}
\end{frame}
\begin{frame}[fragile,label=sec-4-3-4]{雛形のbuild}
 \begin{itemize}
\item \texttt{*.gemspec} のTODOを外す(内容をきちんと書く)
\item \texttt{rake build} でbuildできる
\end{itemize}
\end{frame}
\begin{frame}[fragile,label=sec-4-3-5]{\texttt{executable} の作成と注意}
 \begin{itemize}
\item \texttt{bundle gem -b} で実行可能なスクリプトの雛形ができる
\item 実行属性を \texttt{chmod a+x} でつける
\item \texttt{bundle install -{}-binstubs} を実行すると,上書きされるので注意
\begin{itemize}
\item 手動で他のディレクトリ( \texttt{/exe} )に作成するほうが良い
\end{itemize}
\end{itemize}
\end{frame}
\begin{frame}[fragile,label=sec-4-3-6]{Gemfileとgemspecの関係}
 \begin{itemize}
\item Gemの依存関係のかき分け
\begin{itemize}
\item Gemfile
\begin{itemize}
\item \texttt{Gemfile} に \texttt{gemspec} メソッドがあれば, \texttt{.gemspec} 内のGem依存関係を解決する
\end{itemize}
\item *.gemspec
\begin{itemize}
\item Gemを利用時に必要なGemを追加
\end{itemize}
\end{itemize}
\end{itemize}
\end{frame}
\section{演習課題}
\label{sec-4-4}
\begin{frame}[label=sec-4-4-1]{4-1 Bundlerで簡単なGemを作ってみる}
\begin{block}{課題}
\begin{itemize}
\item 簡単な計算を行うコマンドをGemとして作成しなさい.
\end{itemize}
\end{block}

\begin{block}{例}
\begin{itemize}
\item 生年月日と今の年月日から年齢を計算する
\item 身長と体重を入力して,BMIを出す
\item その他,各自で考えよ
\end{itemize}
\end{block}
\end{frame}

\begin{frame}[fragile,label=sec-4-4-2]{4-2 作成したGemをGitHubで公開する}
 \begin{block}{課題}
\begin{itemize}
\item 作成したGemのソースコードをGitHubに公開しなさい
\begin{itemize}
\item \texttt{git add} , \texttt{git commit} でコミット
\item \texttt{hub create} でGitHubのリポジトリを作成
\item \texttt{git push -u origin master} でGitHubに登録
\end{itemize}
\end{itemize}
\end{block}

\begin{block}{提出}
\begin{itemize}
\item GitHubのURLをLMSに提出する
\end{itemize}
\end{block}
\end{frame}
% Emacs 24.3.2 (Org mode 8.2.5h)
\end{document}