% Created 2014-09-29 月 16:36
\documentclass[t, aspectratio=169]{beamer}
\usepackage{zxjatype}
\usepackage[ipa]{zxjafont}
\setbeamertemplate{navigation symbols}{}
\hypersetup{colorlinks,linkcolor=,urlcolor=gray}
\AtBeginPart
{
  \begin{frame}<beamer|handout>
    \date{\insertpart}
    \maketitle
  \end{frame}
}
\AtBeginSection[]
{
  \begin{frame}<beamer>
  \tableofcontents[currentsection,currentsubsection]
  \end{frame}
}

\usepackage{minted}
\institute[AIIT]{産業技術大学院大学(AIIT)}
\usetheme{Berkeley}
\usecolortheme{seahorse}
\useinnertheme{rectangles}
\author{中鉢 欣秀}
\date{}
\title{フレームワーク開発特論}
\hypersetup{
  pdfkeywords={},
  pdfsubject={},
  pdfcreator={Emacs 24.3.1 (Org mode 8.2.6)}}
\begin{document}

\maketitle


\part{第1章 ガイダンス}
\label{sec-1}
\section{未整理}
\label{sec-1-1}
\begin{frame}[label=sec-1-1-1]{自己紹介}
\end{frame}

\begin{frame}[label=sec-1-1-2]{仮想化環境利用}
\begin{itemize}
\item Rubyの開発環境
\begin{itemize}
\item enPiTの仮想環境を利用
\end{itemize}
\item インストールの方法
\begin{itemize}
\item 動画を参照
\end{itemize}
\end{itemize}
\end{frame}

\begin{frame}[label=sec-1-1-3]{スケジュール}
\begin{itemize}
\item スケジュール
Gitの使い方を先にやる
\end{itemize}
\end{frame}


\part{RubyGemsの概要と周辺のツール群}
\label{sec-2}

\begin{itemize}
\item 初回は開発環境のインストール
\begin{itemize}
\item 開発環境は,enPiT環境を利用する
\end{itemize}
\item コンポーネントとGemの関係
\begin{itemize}
\item 他の言語だとどうなるか?
\end{itemize}
\item 簡単なgemをダウンロードして中身を見てみる
\item Gitの使い方
\begin{itemize}
\item Gemを作るのに,gitは必須
\item bundle gem hoge すると gitのレポジトリができるので
\end{itemize}
\item 演習
\begin{itemize}
\item Gemを生成する
\begin{itemize}
\item 生成された内容を調べる
\item Rakeの使い方
\end{itemize}
\item その後,テストの話をする
\end{itemize}
\end{itemize}

\part{Rubyによる単体テスト}
\label{sec-3}

\part{Rubyによる統合テスト}
\label{sec-4}

\part{GitHubによるコードの共有}
\label{sec-5}

\part{CIツールによるテスト自動化}
\label{sec-6}

\part{まとめ}
\label{sec-7}
% Emacs 24.3.1 (Org mode 8.2.6)
\end{document}